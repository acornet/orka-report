\section{Providing source-line level energy estimates} 
\label{sec:sourceline}

The key extension to Orka was to allow meaningful accounting for API 
calls occurring in subroutines and provide this at the source-line level 
rather than the method-level as was the case previously. To this end, 
\orka{} now maintains an in-memory version of the call stack while it 
replays the logs and attributes any API calls to all the routines in the 
call stack. Therefore, \Orka{} tallies the calls occurring 
\textit{during} the execution of this routine, possibly in a subroutine. 
As estimates given in Joules are hard to interpret, these results will 
be given as the percentage of the total routine cost. Taking 
control-flow into account, the cost of a line of code was defined as the 
cost of a single execution of this line multiplied by the average number 
of times it was executed per call. \Orka{} approximates the energy usage 
of a routine with the cost of its calls to the Android API and assumes 
that the remaining parts of the code have a marginal energy footprint. 
On top of tallying the number of API calls during a routine execution, 
\Orka{} needs to know where in the source code each of these calls 
occurred. This information is provided by the \smali{} \texttt{.line} 
instruction, which indicates what exact line in the source code 
corresponds to the bytecode instructions that follow. Since all the API 
calls happening during the execution of a subroutine are now attributed 
to the parent routine as well, for a given routine, all API calls 
happening during the execution of a subroutine will be attributed to the 
line from which the subroutine is invoked.

%Ultimately, \Orka{} is able to reconstruct the source-code of each 
%routine and to include the source-line estimates. As developers provide 
%only the \apk{} archive of their application, \Orka{} doesn't have 
%access to the original source code but only to its \smali{} equivalent. 
%Reverse engineering bytecode is a difficult problem and is outside the 
%scope of this work. However, the \smali{} bytecode contains enough 
%information to partially reconstruct the source-code of a routine by 
%including only the line for which \Orka{} computes energy-estimates, 
%i.e.\ the lines containing API and subroutines invocations. For each of 
%these source-lines, \Orka{} will be able to display the line number 
%alongside the energy-significant calls contained in that line and the 
%corresponding energy estimates.

\subsection{Implementation}

\subsubsection{Injector}

The first step was to allow the instrumented application to log 
\texttt{(apiName, lineNumber)} pairs. On finding the \smali{} 
\texttt{.line} statement, the injector now updates a variable containing 
the source line-number corresponding to the current instructions. On 
finding an API invocation, this information is then appended to the 
API's name and passed to the \texttt{APILog} function. At this point, 
\Orka{} is able to know where the API was called in a method's body and 
to detect when the API was called from a subroutine, but not to map this 
cost to a specific line in the code of the parent routine. In order to 
access that information, the next step was to allow the instrumented 
application to log messages indicating subroutine invocations alongside 
the line from which the subroutine was called. On finding an 
\texttt{invoke} statement, the injector now checks whether the called 
routine is user-defined and injects a call to a logging function if so. 
To answer the need for more various log messages, the \texttt{APILog} 
function was extended into the \texttt{customLog} function able to log 
custom messages.

This design led log messages to be inserted before the invocation of any 
subroutine, even the non-injected ones. This issue was noticed as the 
\logcat{} results were containing many subroutine invocation messages 
not followed by an entering statement. Therefore, this implementation 
was breaking one main requirement of the injector, namely to enforce 
that the minimum amount of code is injected. To mitigate this issue, the 
injector needs to know which routines are injected prior to the 
injection and add log messages only before calls to injected 
subroutines. To this end, a preparatory phase was added to the injector 
workflow: injected files are now scanned prior to the injection in order 
to build the set of injected methods.

\subsubsection{Analyser and \texttt{routine} class}

The \texttt{routine} class used to view the cost of a routine as the sum 
of the cost of all its API calls. In order to implement fine-grained 
energy feedback, it now more convenient to see the cost of a routine as 
the sum of the cost of all its lines. The cost of a line is defined as 
the cost of the API calls occurring at this line. Finally, the 
\texttt{routine} class was extended in order to generate a reconstructed 
source code including energy estimates from the API data. A typical 
output is presented in Figure \ref{fig:sourcelinefeedback}.

\begin{figure}
\centering
\begin{lstlisting}
method RecList.onCreate, Average cost: 0.0197754308, Calls: 1.0
      2.64% l-1
      1.45% l192    calling RecList.InitializeRepeatButton
      0.88% l227    calling RecList.InitializeThemedButtonsBackgrounds
      0.57% l229    calling Preferences.getCurrentlyPlayingFilePath
     29.04% l247    calling RecList.updateSongList_extended
     21.80% l257    calling android.os.Bundle.getString
     21.80% l258    calling android.os.Bundle.getString
     21.80% l260    calling android.os.Bundle.getString
\end{lstlisting}
\caption{Example of reconstructed source code with fine-grained guidance}
\label{fig:sourcelinefeedback}
\vspace{-0.32in}
\end{figure}

To store from which line a subroutine is called, a second stack was 
added alongside the call stack. On finding a subroutine invocation log, 
the line number from which the subroutine is called is added to the 
stack. On finding an exiting statement, the last line number is popped 
from the stack. By merging these two stacks, we obtain a stack of pairs 
\texttt{(routineName, lineNumber)} indicating which line the API
calls should be attributed to in a given routine. Finally, the names of 
subroutines are stored in a separate dictionary, the keys of which are 
line numbers.

In some specific cases, e.g. a class accessing a static variable of 
another class, injected constructors are called implicitly. As \Orka{} 
is not yet able to detect such cases, subroutine invocation statements 
are sometimes missing and there is no information regarding the line 
number corresponding to the call. However, by comparing the two stack 
sizes we are able to detect this and a default value, as necessary is 
added to the stack to indicate the absence of any information.
