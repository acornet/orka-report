\section{Conclusions and future work}
\label{sec:conc}

In this paper, we presented one of the early software-based approaches 
to providing fine-grained energy feedback (at the source-line level) to 
developers enabling them to investigate energy bugs effortlessly. This 
work is also able to map the energy drain caused by the Wi-Fi antenna 
back to the code and to partially account for the tail-energy. One of 
the limitation of \orka{} is its heavy reliance on the cost of Android 
APIs found by \cite{linares2014mining}. In order for accurate feedback, 
\orka{} would need to have access to the energy estimate of newer APIs. 
Based on its main assumption, \orka{} assumes that the cost of the 
routines making no calls to the Android API is marginal, and it 
consequently provides energy estimates exclusively for routines 
including API invocations. However, a routine may not make any calls to 
the Android API, but instead invoke a subroutine which includes these. 
To this end, the injector should build a call graph of the user-defined 
routines, the leaves and nodes of which would be respectively the API 
calls and the routines.  Moreover, the energy estimates generated by 
monitoring the energy-activity of the hardware should be included in the 
energy estimates provided by \orka{} at the method-level. Finally, this 
paper focussed exclusively on Wi-Fi and should include all other 
hardware components.
