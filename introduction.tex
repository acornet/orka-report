\section{Motivation}

Mobile technologies have become ever-present in our daily lives and many 
people's personal and professional interactions now depend on their 
smartphone or tablet. The power consumption of mobile devices obviously 
grew with the duration of their usage and the complexity of hardware and 
software they involve. By nature, these devices are used away from power 
sources and battery has thus become a critical component for the user 
experience. However, battery technology hasn't managed to meet these new 
requirements and hence energy efficiency has become a major concern of 
users who are now looking for feedback to understand how applications 
drain the battery of their devices \cite{heikkinen2012energy}. Moreover, 
research has shown that implementations perceived as energy greedy will 
receive lower ratings from users \cite{jabbarvand2015ecodroid}. The 
ability to build energy efficient software consequently awards 
developers with a competitive advantage on the market. However, energy 
optimisation is often counter intuitive to many developers and there is 
no real guidance. For example, there is no clear correlation between 
energy and time efficiency \cite{bunse2009choosing}, and time 
optimisation is thus not always useful to reduce the energy footprint of 
a software. Also, some popular design principles such as the decorator 
pattern have bad energy efficiency \cite{sahin2012initial}.

Researchers have consequently developed tools to provide guidance to 
developers by profiling the energy drain of their code. These tools were 
initially tied to specific and expensive power measurement platforms 
which inherently limited their use. Software-based approaches were 
introduced so that energy profiling techniques are accessible to the 
vast majority of developers by placing on them no hardware requirements. 
The first key contribution of this work is to provide source-line level 
energy estimates to the developers. This is achieved by extending Orka 
(one of the first software-based energy profilers) 
\cite{westfield2016orka}.

To provide accurate and meaningful energy feedback however, energy 
profilers should also take into account the drain caused by various 
hardware components. This means being able to map the hardware energy 
usage back to the code with the finest granularity possible. Moreover, 
mobile devices exhibit several asynchronous power behaviours, most 
significant of which is \textit{tail-energy}, which proves challenging 
for the development of energy profilers 
\cite{pathak2012energy,li2013calculating}. A component is said to 
exhibit tail-power behaviour if it stays in a high-power for a constant 
time after processing a workflow and this phenomenon can play an 
important role in the energy drain. Thus far, only a few contributions, 
such as \eprof{} \cite{pathak2012energy} have focussed on tail-energy by 
relying on hardware- or model-based approaches. Therefore, no tool 
accounting for tail-energy was accessible to the majority of developers. 
To allow for this, the second major contribution of this work is for our 
software-based solution to also provide tail-energy accounting. We have 
focussed exclusively on Wi-Fi for now and will in future include other 
hardware components.

The remainder of this paper is organised as follows: Section 
\ref{sec:background} outlines related work, Sections 
\ref{sec:sourceline} and \ref{sec:hw} detail the major contributions. 
Section \ref{sec:eval} provides an evaluation, and Section 
\ref{sec:conc} concludes the paper and provides ideas for future work.

